\documentclass[12pt]{article}%
\usepackage{amsfonts}
\usepackage{hyperref}
\usepackage{fancyhdr}
\usepackage{comment}
\usepackage{mathrsfs}
\usepackage{mathtools}
\usepackage{graphicx}
\usepackage[a4paper, top=2.5cm, bottom=2.5cm, left=2.2cm, right=2.2cm]%
{geometry}
\usepackage{times}
\usepackage{amsmath}
\usepackage{amsthm}
%\usepackage{changepage}
\usepackage{amssymb}
%\usepackage{graphicx}%
\setcounter{MaxMatrixCols}{30}
\newtheorem{theorem}{Theorem}
\newtheorem{acknowledgement}[theorem]{Acknowledgement}
\newtheorem{algorithm}[theorem]{Algorithm}
\newtheorem{axiom}{Axiom}
\newtheorem{case}[theorem]{Case}
\newtheorem{claim}[theorem]{Claim}
\newtheorem{conclusion}[theorem]{Conclusion}
\newtheorem{condition}[theorem]{Condition}
\newtheorem{conjecture}[theorem]{Conjecture}
\newtheorem{corollary}[theorem]{Corollary}
\newtheorem{criterion}[theorem]{Criterion}
\newtheorem{definition}[theorem]{Definition}
\newtheorem{example}[theorem]{Example}
\newtheorem{exercise}[theorem]{Exercise}
\newtheorem{lemma}[theorem]{Lemma}
\newtheorem{notation}[theorem]{Notation}
\newtheorem{problem}[theorem]{Problem}
\newtheorem{proposition}[theorem]{Proposition}
\newtheorem{remark}[theorem]{Remark}
\newtheorem{solution}[theorem]{Solution}
\newtheorem{summary}[theorem]{Summary}

\newcommand{\Q}{\mathbb{Q}}
\newcommand{\R}{\mathbb{R}}
\newcommand{\C}{\mathbb{C}}
\newcommand{\Z}{\mathbb{Z}}

\newcommand{\E}{\mathrm{E}}
\newcommand{\Var}{\mathrm{Var}}
\newcommand{\Cov}{\mathrm{Cov}}

\begin{document}

\title{MATH618 Homework 5}
\author{Fernando}
\date{\today}
\maketitle
\section*{Problem 1}
\subsection*{$L_1+L_2$ is compact}
\textbf{Proof:}\\
Let $x_k$ be a bounded sequence. Since $L_1$ is compact we can find a
subsequence $x_{k_i}$ such that $L_1(x_{k_i})$ converges. Because $x_{k_i}$ is
also bounded we can find $x_{k_{i_j}}$ such that $L_2(x_{k_{i_j}})$ converges
(because $L_2$ is compact), and thus $(L_1+L_2)(x_{k_{i_j}})$ converges.\\
\textbf{Conclusion:} any bounded sequence under $L_1+L_2$
has a convergent subsequence. $\qedsymbol$
\subsection*{$\alpha L$ is compact}
\textbf{Proof:}\\
Let $x_k$ be a bounded sequence. Since $L$ is compact we can find a subsequence
$x_{k_i}$ such that $L(x_{k_i})$ converges, then clearly $\alpha
L(x_{k_i})$ is also convergent, hence $\alpha L$ is compact. $\qedsymbol$
\section*{Problem 2}
\textbf{Proof:}\\
Let $x_k$ be a bounded sequence. Since $A$ is compact we can find a
subsequence $x_{k_i}$ such that $A(x_{k_i})$ converges. Because $x_{k_i}$ is
also bounded we can find $x_{k_{i_j}}$ such that $B(x_{k_{i_j}})$ converges
(because $B$ is compact), and thus $AB(x_{k_{i_j}})$ converges. $\qedsymbol$
\section*{Problem 3}
\subsection*{Part a}
This operator is bounded because of Parseval's identity, i.e.
\[
	||f||_X = C ||Lf||_Y
\]
for some constant $C$. So any bounded sequence will be mapped to a
bounded sequence.
\subsection*{Part b}
This operator is bounded because of Parseval's identity, i.e.
\[
	||f||_Y=C||L^{-1}f||_X
\]
for some constant $C$. So any bounded sequence will be mapped to a
bounded sequence.
\subsection*{Part c}
$L$ is not compact because if we take $\{x_k\}$ where $x_k=\sin(kt)$ (bounded)
this sequences gets mapped to $\{e_k\}$ which we know is not precompact.

$L^{-1}$ is not compact basically for the same reason, $\{e_k\}$ (bounded) gets
mapped to $\{\sin(kt)\}$ which again we know is not precompact.
\section*{Problem 4}
We want to estimate (find a bound for)
\[
	A_k
	=\frac{||y_k-\tilde{y}_k||}{||y_k||}\bigg/\frac{||x_k-\tilde{x}_k||}{||x_k||}
	=\frac{||y_k-\tilde{y}_k||\cdot||x_k||}{||y_k||\cdot||x_k-\tilde{x}_k||}
	=\frac{||y_k-\tilde{y}_k||}{||x_k-\tilde{x}_k||} \cdot
	\frac{||x_k||}{||y_k||}
	=\frac{||L(\delta x_k)||}{||\delta x_k||} \cdot \frac{||x_k||}{||L(x_k)||}
\]
Then
\[
A_k=\frac{||L(\delta x_k)||}{||\delta x_k||} \cdot \frac{||x_k||}{||L(x_k)||}\leq
||L||\cdot \frac{||x_k||}{||L(x_k)||}
\]
For simplicity let's assume that $L$ is invertible (otherwise I am not sure how
to proceed), then we have
\[
A_k\leq
||L||\cdot \frac{||L^{-1}(x_k)||}{||x_k||} \leq ||L|| \cdot ||L^{-1}||.
\]
If $L$ was not bounded then $A_k \to \infty$ as $k \to \infty$ and that means
that small errors in our measurements will be amplified greatly as we increase
the dimension. This means that our end result \textbf{could} be completely
useless. That is why we need $L$ to be bounded.

In regards to $L^{-1}$ there is a theorem (bounded inverse theorem) that
ensures that under appropriate conditions if $L$ is a bijective bounded linear
operator then $L^{-1}$ is also bounded.
\section*{Problem 5}
\end{document}
